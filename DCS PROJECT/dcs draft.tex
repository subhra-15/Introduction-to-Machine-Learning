\documentclass{article}
\usepackage[affil-it]{authblk}
\usepackage{graphicx}
\usepackage[english]{babel}
\usepackage[utf8x]{inputenc}
\usepackage{amsmath}
\usepackage{graphicx}
\usepackage[colorinlistoftodos]{todonotes}

\begin{document}

\begin{titlepage}

\newcommand{\HRule}{\rule{\linewidth}{0.5mm}} % Defines a new command for the horizontal lines, change thickness here

\center % Center everything on the page
 
%----------------------------------------------------------------------------------------
%	HEADING SECTIONS
%----------------------------------------------------------------------------------------

\textsc{\LARGE Indira Gandhi Delhi Technical University for Women}\\[1.5cm] % Name of your university/college
\textsc{\Large Digital Circuits and Systems}\\[0.5cm] % Major heading such as course name
\textsc{\large Project}\\[0.5cm] % Minor heading such as course title

%----------------------------------------------------------------------------------------
%	TITLE SECTION
%----------------------------------------------------------------------------------------

\HRule \\[0.4cm]
{ \huge \bfseries Password Security System}\\[0.4cm] % Title of your document
\HRule \\[1.5cm]
 
%----------------------------------------------------------------------------------------
%	AUTHOR SECTION
%----------------------------------------------------------------------------------------

\begin{minipage}{0.4\textwidth}
\begin{flushleft} \large
\emph{Author:}\\
Subhra Sarita Sahoo \textsc{(00601022018)}\\
Somya Gandhi \textsc{(03201022018)} % Your name
\end{flushleft}
\end{minipage}
~
\begin{minipage}{0.4\textwidth}
\begin{flushright} \large
\emph{Supervisor:} \\
Dr. Jasdeep Kaur  \textsc{Dhanoa} % Supervisor's Name
\end{flushright}
\end{minipage}\\[2cm]

% If you don't want a supervisor, uncomment the two lines below and remove the section above
%\Large \emph{Author:}\\
%John \textsc{Smith}\\[3cm] % Your name

%----------------------------------------------------------------------------------------
%	DATE SECTION
%----------------------------------------------------------------------------------------

{\large \today}\\[1cm] % Date, change the \today to a set date if you want to be precise

%----------------------------------------------------------------------------------------
%	LOGO SECTION
%----------------------------------------------------------------------------------------

\includegraphics[scale=0.40]{logo.jpg}\\[1cm] % Include a department/university logo - this will require the graphicx package
 
%----------------------------------------------------------------------------------------

\vfill % Fill the rest of the page with whitespace

\end{titlepage}


\tableofcontents
\newpage
\section{Introduction}
Science and technology is a very essential part of our life. It has given us an immense amount of inventions that we use in our day to day lives. Without these creative innovations and technological developments, life would be extremely tough as compared to what it is now. Password security system is one of these discoveries. A password is a string of characters that is used to check the identity of a person or to gain access to that particular site or system. It is supposed to be kept a secret from those who are not allowed entry into that system.
The use of passwords helps us to protect sensitive and private information. These are used in mobile phones, Internet sites, social media, ATMs, etc.
\subsection{History}
Passwords have been in use since a very long time. People in early days would ask for a specific password or watchword and only allow access into that place if the person knew that word. This was the earliest usage of password. Since then, passwords have been used in computing. The first computer system to use a password was the Compatible Time-Sharing System (CTSS) which was a operating system introduced at Massachusetts Institute of Technology in 1961. The existence of combinational locks dates back to the year 1200's. Combinational locks are basically locking devices that can take up a sequence of numbers or characters and these characters when inserted in it can help in opening the lock. The first commercially-viable single-dial combination lock was patented on 1 February 1910 by John Junkunc, owner of American Lock Company. These devices offer security to the claimant from outsiders who want to access personal property. Password based locking system is one of the modern electronic lock system.
\subsection {Aim And Theory}
This project aims to create a four character lock wherein four key switches exist which hold the correct password to unlock the lock. There would be another four data entry switches where the person who wants to open this lock can insert the password. If the password is correct, a green led lights up and the lock opens whereas if the password is incorrect, a red led lights up which can help in alarming the owner of an intruder. This device that is created can be used in homes as well as organisations.
\newpage
\section{Project Details}
\subsection{Equipments Needed}
\begin{itemize}
\item 74HC02 Quad NOR Gate
\item 74HC86 Quad XOR Gate
\item 2 Four position DIP Switch
\item 2 Light Emitting Diode
\item 4 Switching Diode
\item 10 Ten kilo ohm resistor 
\item 2 Four hundred seventy kilo ohm resistor
\item Battery
\item Breadboard
\item Connecting wires 
\end{itemize}
\subsection{Description About Component}
\subsubsection {74HC02 Quad NOR Gate}
\begin{itemize}
\item This integrated circuit is used to implement logical operation NOR.A high output will results if both the input is low, if one or both the input is high the output will be low.
\item The truth table and symbol for NOR gate is as fallow:
\end {itemize}
\center
\includegraphics[scale=0.55]{dcs2.png}
\begin{itemize}
\item This gate is called universal gate because NOR gate can be combined to form any kind of gate.
\item The pin diagram of 74HC02 QUAD NOR gate is as fallow:
\end{itemize}
\center
\includegraphics[scale=0.55]{dcs1.jpg}
\center
\begin{itemize}
\item 01- output 1
\item 02- input 1A
\item 03- input 2A
\item 04- output 2
\item 05- input 2A
\item 06- input 2B
\item 07- ground
\item 08- input 3A
\item 09- input 3B
\item 10- output 3
\item 11- input 4A
\item 12- input 4B
\item 13- output 4
\item 14- power
\end{itemize}
\subsubsection{74HC86 Quad XOR Gate}
\begin{itemize}
\item This integrates circuit is used to implement logical operation XOR.It is known as Exclusive Or gate.A high output will results if the number of high input is odd, otherwise output will be low.
\item The truth table and symbol for XOR gate is as fallow:
\end {itemize}
\center
\includegraphics[scale=0.55]{dcs10.jpg}
\begin{itemize}
\item The pin diagram of 74HC86 QUAD XOR gate is as fallow:
\end{itemize}
\center
\includegraphics[scale=0.55]{dcs4.jpg}
\begin{itemize}
\item 01- input 1A
\item 02- input 1B
\item 03- output 1
\item 04- input 2A
\item 05- input 2B
\item 06- output 2
\item 07- ground
\item 08- output 3
\item 09- input 3B
\item 10- input 3A
\item 11- output 4
\item 12- input 4A
\item 13- input 4B
\item 14- power
\end{itemize}
\subsubsection{Four Position DIP Switch}
\begin{itemize}
\item Here DIP stands for Dual Inline Package switch.
\item A dual in-line package switch, or DIP switch, is actually set of small manual electronic switches that are designed to be packaged with other circuits. The term DIP switch may refer either to an individual switch on a multi-switch unit, or to the entire unit as a whole.
\item In this four position DIP switch there are four switches packed together.
\item In this project these switches are used as the key code switch which holds the actual password and data entry switch which allows the user to enter the password
\end{itemize}
\center
\includegraphics[scale=0.55]{dcs5.jpg}

\subsubsection{Switching Diode}
\begin{itemize}
\item Switching diode is used for switching of the small signal that is upto 100mA.
\item This diode is also calles Pulse diode and Schottky diode.It is used in pulse detector.
\item Switching diode can be operates as rectifier, transient voltage supressor or detection diode.
\end{itemize}
\center
\includegraphics[scale=0.55]{dcs11.jpg}

\subsubsection{Resistors}
\begin{itemize}
\item A resistor is passive two terminal electric component the add resistance to the circuit.Resistors can be used to reduce the current flow, divide voltage, adjust signal level and terminate transmission line.
\end{itemize}
\center
\includegraphics[scale=0.60]{dcs13.jpg}
\begin{itemize}
\item Resistors are marked with color coded bands that have a different number per color, as well as the tolerance of the resistor. With the help of the resistor color chart you can determine the value and resistance.
\end{itemize}
\center
\includegraphics[scale=0.50]{dcs12.png}

\subsubsection{Battery}
\begin{itemize}
\item Group of electrochemical cells with external connection for powering electrical devices such as flashlight, mobile phones etc.
\item There are two terminals in a battery,the positive terminal is called cathode and negative terminal is called anode. The negative terminal is the source of electron that will flow through the entire circuit to positive terminal. 
\item When a battery is connected to an external electric load, a redox reaction converts high-energy reactants to lower-energy products, and the free-energy difference is delivered to the external circuit as electrical energy.
\end{itemize}
\includegraphics[scale=0.50]{dcs14.jpg}

\subsubsection{Bread Board}
\begin{itemize}
\item A breadboard is a rectangular pastic board with a bunch of tiny holes on it. These holes allows to insert the electrical components like resistor, battery, switches etc.
\item These connections are not permanent and these can be easily removed.The leads can fit into the breadboard because the inside of a breadboard is made up of rows of tiny metal clips.
\item There are some numbers, letters, plus and minus sign written on the breadboard, the general purpose of these are to locate the holes on the bread board.
\item The following diagram shows how the cells in the breadboard are connected:
\end{itemize}
\includegraphics[scale=0.50, angle=90]{dcs15.png}

\section{THE UNDERSTANDING OF BASIC CIRCUIT DIAGRAM}
\includegraphics{dcs8.png}
\begin{itemize}
\item KEY CODE SWITCH: This is four position DIP switch that contains the password
\item DATA ENTRY SWITCH: This four position DIP switch allows the user to enter the password.
\item This circuit illustrate the use of XOR gate as comparator. In this XOR gate compare the four bit of binary number, each number is entered to the circuit though switches.
\item If the two code matches completely then the green led glows and if one of the bit doesn't match then the  red led glows.
\item This circuit can be used as an alarming system in home instead of connecting red led if we use an alarm system when the wrong code is entered then it will be helpful.
\item The four XOR gates output terminals are connected through a diode network which functions as a four-input OR gate: if any of the four XOR gates outputs a “high” signal—indicating that the entered code and the key code are not identical—then a “high” signal will be passed on to the NOR gate logic.
\item If the two 4-bit codes are identical, then none of the XOR gate outputs will be “high,” and the pull-down resistor connected to the common sides of the diodes will provide a “low” signal state to the NOR logic. 
\item The NOR gate logic performs a simple task: prevent either of the LEDs from turning on if the “Enter” pushbutton is not pressed. Only when this pushbutton is pressed can either of the LEDs energize.
\item If the Enter switch is pressed and the XOR outputs are all “low,” the “Go” LED will light up, indicating that the correct code has been entered. If the Enter switch is pressed and any of the XOR outputs are “high,” the “No go” LED will light up, indicating that an incorrect code has been entered.
\center
\includegraphics[scale =0.70]{dcs9.png}
\end{itemize}

\section{CONCLUSION}
\begin{itemize}
\item Every creative work is started such that to aim on a specific motto. The main goal of our project was to create a security system that would provide us with maximum security as well as is pocket friendly. Our project creates a password security lock which is cost effective. It is also very reliable and easy to make. We were able to create that using 74HC02 Quad NOR Gate and 74HC86 Quad XOR Gate.This project help us to understand better how the circuit is built and also the real price of every components. There is also still a lot to improve in this project as we consider this project basic and more components can be added up to this circuit. The above made system can be used in ATMs, the banking sector, the IT sector as well as in our homes. With so many applications of this system, this can surely be used for the benefit for a wide array of people.
\end{itemize}
\section{REFERENCES}
\begin{itemize}
\item $ https://www.researchgate.net/publication/325093383 $
\item $https://www.researchgate.net$
\item $https://patents.google.com/patent$
\item $https://www.slideshare.net/FatimaQayyum1/security-system-using-xor-nor$
\item $https://dl.acm.org/doi/epdf$
\item $https://www.sciencebuddies.org/science-fair-projects/references$

\end{itemize}




\end{document}